\documentclass[]{article}

\usepackage{parskip}
\usepackage{url}
\usepackage{amssymb}

\usepackage{listings}
\lstloadlanguages{Haskell}
\lstnewenvironment{code}
    {\lstset{}%
      \csname lst@SetFirstLabel\endcsname}
    {\csname lst@SaveFirstLabel\endcsname}
    \lstset{
      basicstyle=\small\ttfamily,
      flexiblecolumns=false,
      basewidth={0.5em,0.45em},
      literate={+}{{$+$}}1 {/}{{$/$}}1 {*}{{$*$}}1 {=}{{$=$}}1
               {>}{{$>$}}1 {<}{{$<$}}1 {\\}{{$\lambda$}}1
               {\\\\}{{\char`\\\char`\\}}1
               {->}{{$\rightarrow$}}2 {>=}{{$\geq$}}2 {<-}{{$\leftarrow$}}2
               {<=}{{$\leq$}}2 {=>}{{$\Rightarrow$}}2 
               {\ .}{{$\circ$}}2 {\ .\ }{{$\circ$}}2
               {>>}{{>>}}2 {>>=}{{>>=}}2
               {|}{{$\mid$}}1
               {forall}{{$\forall$}}2
               {>>=}{{$\gg\!=$}}2 {>>}{{$\gg$}}2
               {()}{{$\square$}}2
    }

\newcommand{\ignore}[1]{}
\newcommand{\function}[1]{\texttt{#1}}
\newcommand{\type}[1]{\texttt{#1}}

\title{Technical Report: Implementing \& Designing the Deco Programming Language}
\author{Andy Kitchen}

\date{2011-10-03}

\begin{document}

\maketitle

\section{Overview}

The programming language Deco is a impure functional programming language.
this report discusses the design process of Deco and it's practical
implementation in Haskell.

The task of building an interpreter for an impure~functional~language inside
a~pure~one is an interesting task. Putting great stress on the abstraction
used to represent state and interaction with the outside world.

While implementing Deco several high-level and distinctive features of
functional programming have been used. Including monad~transformers,
parser~combinators, rank-2~types and delimited~continuations. They have also
been made to work in unison, leading to an interesting synthesis. High-level
features such as these are often described from a very theoretical perspective
and in isolation. This report aims to present the theoretical and the
practical together; and explore the interactions of different features.

\pagebreak
\input{../Evaluate.lhs}

% \section{Main}
% \input{../Main.lhs}
%
% \section{Lexer}
% \input{../Lexer.lhs}
%
% \section{Parser}
% \input{../Parser.lhs}
%
% \section{Primitives}
% \input{../Primitives.lhs}

%\nocite{*}
%\bibliographystyle{plain}
%\bibliography{pcp}
\end{document}
